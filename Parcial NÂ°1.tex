\documentclass{article}
\usepackage[utf8]{inputenc}
\usepackage[spanish]{babel}
\usepackage{listings}
\usepackage{graphicx}
\graphicspath{ {images/} }
\usepackage{cite}

\begin{document}

\begin{titlepage}
    \begin{center}
        \vspace*{1cm}
            
        \Huge
        \textbf{Parcial N° 1}
            
        \vspace{2 cm}
        \LARGE
    
            
        \vspace{2 cm}
            
        \textbf{Ivonne  Rosero }\\
        \large
        1007687589
        
        \vspace{2cm}
        \LARGE
        
        \textbf{Rigoberto Berrio}\\
        \large
        1040327583
            
        \vfill
            
        \vspace{0.8cm}
            
        \Large
        Despartamento de Ingeniería Electrónica y Telecomunicaciones\\
        Universidad de Antioquia\\
        Medellín\\
        Abril de 2021
            
    \end{center}
\end{titlepage}

\tableofcontents
\newpage
\section{Análisis del problema y consideraciones para la alternativa de solución propuesta.}\label{intro}
 
Gracias a los apuntes y grabaciones de clase se estan haciendo pruebas para afianzar el conocimiento para poder asimilarlo  al nivel del examen.

Decidimos utilizar 8 integrados para que cada uno controlara 8 leds y asi completar los 64 leds.


\section{Esquema que describa las tareas que se definieron en el desarrollo del algoritmo.}\label{intro}




\section{Esquema donde describa las tareas que se definieron en el desarrollo del algoritmo.}\label{intro}

\section{Problemas de desarrollo.}\label{intro}

\section{Evolución del algoritmo y consideraciones a tener en cuenta en la implementación.}\label{intro}





\end{document}
